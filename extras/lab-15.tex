\documentclass{tufte-handout}

\usepackage{xcolor}

% set image attributes:
\usepackage{graphicx}
\graphicspath{ {images/} }

% set hyperlink attributes
\hypersetup{colorlinks}

% create environment for bottom paragraph:
\newenvironment{bottompar}{\par\vspace*{\fill}}{\clearpage}

\usepackage{enumerate}

% set table attributes
\usepackage{tabu}
\usepackage{booktabs}

% ============================================================

% define the title
\title{SOC 4015/5050: Lab-15 - Chi-Squared}
\author{Christopher Prener, Ph.D.}
\date{Fall 2018}

% ============================================================

\begin{document}

% ============================================================

\maketitle % generates the title

% ============================================================

\vspace{5mm}
\section{Directions}
Please complete all steps below. All work should be uploaded to your GitHub assignment repository by 4:00pm on Monday, December 17\textsuperscript{th}, 2018. All data can be obtained from the \texttt{testDriveR} package's \texttt{auto17} data set.

\vspace{5mm}
\section{Analysis Development}
Using RStudio and your operating system's file manager, create an R Project in the \textit{existing} directory in your assignments repository named \texttt{Lab-14}. Add a \texttt{README.md} file, notebook, and all necessary folders before beginning.\sidenote{This initial section follows the project workflow that is available in the \texttt{lecture-03} repo!}

\vspace{5mm}
\section{Data Preparation}
\begin{enumerate}
\item Create a new logical variable that is \texttt{TRUE} if the vehicle is a ``German'' vehicle (i.e. one made by BMW, Mercedes, Porsche, and Volkswagen) and \texttt{FALSE} otherwise.
\item Subset your data so that it contains only the \texttt{id}, your new logical variable, and the \texttt{driveStr} variables.
\end{enumerate}

\vspace{5mm}
\section{Create Tables}
Using the data created in Part 1, answer the following questions. 
\begin{enumerate}
\setcounter{enumi}{2}
\item Create a two-way table of the logical variable you created above and \texttt{driveStr} using \texttt{janitor} that includes 
\begin{enumerate}
\item a total row at the bottom and a total column,
\item properly formatted row percents that are display three decimal places,
\item and frequency values in the ``front'' position.
\end{enumerate}
\end{enumerate}

\vspace{5mm}
\section{Fit the Chi-Square and Check Assumptions} 
Using the data created in Part 1, answer the following questions. 
\begin{enumerate}
\setcounter{enumi}{3}
\item Fit and interpret the results of a chi-squared test comparing the relationship of German vehicles to drivetrains. Is there a meaningful relationship between these two variables?
\item Does this model violate the Cochran conditions assumption?
\item Regardless of your answer above, fit a Fisher's Exact Test on these same data. Does this change your interpretation of question 6?
\end{enumerate}

% ============================================================
\end{document}